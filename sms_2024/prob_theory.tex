\documentclass{article}
\usepackage{graphicx} % Required for inserting images
\usepackage{cmap}  % should be before fontenc
\usepackage[T2A]{fontenc}
\usepackage[utf8]{inputenc}
\usepackage[russian]{babel}
\usepackage{amsmath,amssymb,amsthm}
\usepackage{unicode-math}
\usepackage[pdftex,colorlinks=true,linkcolor=blue,urlcolor=red,unicode=true,hyperfootnotes=false,bookmarksnumbered]{hyperref}
\usepackage[nameinlink]{cleveref}
\usepackage{indentfirst}

\newcommand{\E}{\ensuremath{\mathsf{E}}}  
\newcommand{\D}{\ensuremath{\mathsf{D}}}  
\newcommand{\Prb}{\ensuremath{\mathsf{P}}}  
\newcommand{\eps}{\varepsilon}  
\renewcommand{\phi}{\varphi} 
\renewcommand{\le}{\leqslant}  
\renewcommand{\leq}{\leqslant}  
\renewcommand{\ge}{\geqslant}  
\renewcommand{\geq}{\geqslant}  
\renewcommand\qedsymbol{Q.E.D.}

\newtheorem{theorem}{Теорема} 
\newtheorem{lemma}{Лемма}
\newtheorem{consequence}{Следствие} 

\theoremstyle{definition}
\newtheorem{definition}{Определение}[section]
\newcommand{\question}{\textbf{Вопрос аудитории: }}  

\theoremstyle{remark}
\newtheorem*{remark}{Примечание}

\Crefformat{lemma}{#2Lemma#3}   
\crefformat{lemma}{#2lemma#3}

\pagestyle{myheadings}
\markright{Probablity Theory. SMS 2024. Greg Sosnovtsev\hfill}  

\begin{document}

\tableofcontents


\section{День 0*. Прелюдия, повторяем комбинаторику}

\section{День 1. Случайные события и элементарное определение вероятности}

Назовем \textit{множеством элементарных исходов} $\varOmega = \{\omega_1\omega_2,...\omega_n \}$ такое конечное множеством, 
что $\omega_i$ и $\omega_j$ несовместны

Тогда \textit{Событие} - это любое множество элементарных исходов. 

Пусть $A \subset \varOmega$ - событие, тогда
\begin{itemize}
    \item $ A \cup B = \{\omega \in \varOmega : \omega \in A$ или $ \omega \in B\} $ 
    \item  $ A \cap B = \{\omega \in \varOmega : \omega \in A$ и $ \omega \in B\} $ 
    \item $ \complement{A} = \{ \omega \in \varOmega : \omega \notin A \} $
\end{itemize}

\begin{remark}

Рассмотрим эксперимент бросок кубика d6 

$\varOmega = \{1,2,3,4,5,6\}$

Событие, что выпала грань с четным числом - \{2,4,6\}
Событие, что выпала грань с числом меньше 3 - \{1,2\}

\end{remark}

Определим некоторую функцию $P: \varOmega \to [0,1]$, которую назовем 
\textit{распределением вероятностей}
Такую, что 
$$
\sum_{\omega \in \varOmega} P(\omega)  = 1
$$

Вероятность P(A) \textit{события} A тогда определим как 
$$
P(A) = \sum_{\omega \in A} P(\omega)  
$$

Пару $(\varOmega, P)$ будем называть \textit{Дискретным вероятностным пространством}



Примитивные свойства:

\begin{enumerate}
 \item $ P(\varnothing) = 0 $
 \item $ P(\varOmega) = 1 $
 \item $P(\complement{A}) = 1 - P(A)$
 \item Если A и B \textit{не совместны} то есть $A \cap B = \varnothing$ то $ P(A \cup B) = P(A) + P(B) $
 \item $ P(A \cup B) = P(A) + P(B) - P(A \cap B) $
\end{enumerate}

\question{Верно ли что нулевая вероятность может быть только у пустого события ($\varnothing$) ?}

\begin{theorem} [Формула включений-исключений]
    $$ 
    P(\bigcup\limits_{i=1}^{m} A_i) = \sum_{1}^{m} P(A_i) - \sum_{i<j} P(A_i \cap A_j) + \sum_{i<j<k} P(A_i \cap A_j \cap A_k) ... + (-1)^{m-1} P(\bigcap\limits_{i=1}^{m} A_i)
    $$
\end{theorem}
\begin{proof}
    Будем вести доказательство индукцией по m

    База при m = 1 очевидна, при m = 2 сошлемся на пятый пункт, который также очевиден.

    Переход от m к m+1:

    Пусть $B = \bigcup\limits_{i=1}^{m} A_i$

    $$
    P(\bigcup\limits_{i=1}^{m+1} A_i) = P(B \cup A_{m+1}) = P(B) + P(A_{m+1}) - P(B \cap A_{m+1})
    $$
    Пусть $B_i = A_i \cap A_{m+1} \Rightarrow P(B \cap A_{m+1}) = P(\bigcup\limits_{i=1}^{m} B_i)$
    $$
    \Rightarrow P(B) + P(A_{m+1}) - P(B \cap A_{m+1}) = P(B) + P(A_{m+1}) - P(\bigcup\limits_{i=1}^{m} B_i) 
    $$

    $$
    = (\sum_{1}^{m} P(A_i)  - \sum_{i<j \le m} P(A_i \cap A_j)  + \sum_{i<j<k \le m} P(A_i \cap A_j \cap A_k) - ... ) + P(A_{m+1}) 
    $$
    $$
    - (\sum_{1}^{m} P(A_i \cap A_{m+1})  - \sum_{i<j \le m} P(A_i \cap A_j  \cap A_{m+1})  + ... )
    $$

    Остается сгруппировать и понять, что это именно то, что нас интересует.
\end{proof}

Рассмотрим важный частный случай, когда все элементарные исходы равновозможны, 
то есть $P(w_i) = \frac{1}{|\varOmega|}$

Тогда вероятность события A равна
$$
P(A) = \frac{|A|}{|\varOmega|}
$$


\section{День 2. Условная вероятность и независимые события}

Пусть $P(B) > 0$ 

Тогда Вероятность события A \textit{при условии} что наступило событие B определим как
$$
P(A | B) = \frac{P(A \cap B)}{P(B)}
$$

Свойства 
\begin{enumerate}
    \item $P(A | A) = 1$ и если $B \subset A$ то $P(A | B) = 1$
    \item $P(\varnothing | B) = 0$
    \item Если $A_1$ и $A_2$ не совместны, то $P(A_1 \cup A_2 | B) = P(A_1 | B) + P(A_2 | B)$
    \item $P(\complement{A} | B) + P(A | B) = 1$
    \question{Будет ли равняться 1 $P(A | B) + P(A | \complement{B})$}
\end{enumerate}

\section{День 3. Случайные величины и вероятностные характеристики}

\section{День 4. Геометрическая вероятность и Метод Монте-Карло}

\section{День 5. Эпилог, что дальше?}

\section{День 6. Зачет}

\section*{Post Scriptum}

\end{document}