\documentclass{article}
\usepackage{graphicx} % Required for inserting images
\usepackage{cmap}  % should be before fontenc
\usepackage[T2A]{fontenc}
\usepackage[utf8]{inputenc}
\usepackage[russian]{babel}
\usepackage{amsmath,amssymb,amsthm}
\usepackage{unicode-math}
\usepackage[pdftex,colorlinks=true,linkcolor=blue,urlcolor=red,unicode=true,hyperfootnotes=false,bookmarksnumbered]{hyperref}
\usepackage[nameinlink]{cleveref}
\usepackage{indentfirst}

\newcommand{\E}{\ensuremath{\mathsf{E}}}  
\newcommand{\D}{\ensuremath{\mathsf{D}}}  
\newcommand{\Prb}{\ensuremath{\mathsf{P}}}  
\newcommand{\eps}{\varepsilon}  
\renewcommand{\phi}{\varphi} 
\renewcommand{\le}{\leqslant}  
\renewcommand{\leq}{\leqslant}  
\renewcommand{\ge}{\geqslant}  
\renewcommand{\geq}{\geqslant}  
\renewcommand\qedsymbol{Q.E.D.}

\newtheorem{theorem}{Теорема} 
\newtheorem{lemma}{Лемма}
\newtheorem{consequence}{Следствие} 

\theoremstyle{definition}
\newtheorem{definition}{Определение}[section]
\newcommand{\question}{\textbf{Вопрос аудитории: }}  

\theoremstyle{remark}
\newtheorem*{remark}{Примечание}

\Crefformat{lemma}{#2Lemma#3}   
\crefformat{lemma}{#2lemma#3}

\pagestyle{myheadings}
\markright{Graph Theory. SMS 2024. Greg Sosnovtsev\hfill}  

\begin{document}

\tableofcontents

\section{День 1. Введение в теорию графов}

Небольшое введение и мотивация, рассказ что будет включать этот курс. В первый день мы введем необхоимый минимум терминов и зададим корень дерева, которое в последующие дни будет расширяться в глубину.

\subsection{Базовые понятия}
\question Что такое граф?


Пусть $G$ — граф. $G = (V(G), E(G))$, где $V(G)$ — множество \textit{вершин} графа $G$,а  $E(G)$ —
множество \textit{ребер} графа G. В этом курсе мы будем рассматривать лишь конечные графы. Количество вершин графа $G$ мы будем обозначать через $v(G)$, а количество ребер — через $e(G)$.
Если не упоминается обратное, граф считается неориентированным,
тогда каждое его ребро имеет два конца, порядок которых не имеет значения. 


Ребро e называется \textit{петлёй}, если начало и конец e совпадают.
Рёбра e и  $e'$ называются \textit{кратными}, если множества их концов совпадают.
Запись e = xy будет обозначать, что вершины x и y — концы ребра e.
В случае, когда граф не имеет кратных рёбер, концы ребра его однозначно задают. Если же кратные рёбра допустимы, возможны несколько
рёбер с концами x и y и запись e = xy допускает наличие другого ребра $e'$ = xy.
В этом курсе мы зачастую будем рассматривать графы без кратных ребер и петель, иначе это будет явно оговорено. 


Про концы ребра e = xy — вершины x и y — мы будем говорить,
что они \textit{соединены} ребром e. Соединённые ребром вершины мы будем
называть \textit{смежными}. Кроме того, мы будем называть смежными рёбра,
имеющие общий конец. Если вершина x — конец ребра e, то мы будем
говорить, что x и e \textit{инцидентны}. Также будем говорить что ыершина $v$ \textit{смежна} множеству вершин $U$, если $v \notin U$ и $U$ содержит вершину смежную c $v$.

\begin{definition}
Для любой вершины $v \in V(G)$  обозначим через  $N_{G}(v)$ множество вершин смежных с $v$ и будем называть \textit{окрестностью} вершины $v$. Аналогично опеределим окрестность множества вершин $N_{G}(U)$
\end{definition}

\begin{definition}
Назовем количество ребер графа $G$ инцидентных вершине $v$ ее \textit{степенью} и дудем обозначать как $d_{G}(v)$. Также за $\delta(G)$ и $\Delta(G)$ обозначим соответственно минимальную и максимальную степени графа $G$ 
\end{definition}

\question На конкретном примере графа покажите описанные выше понятия.

\begin{definition}
Назовем граф из n вершин, в котором проведены ребра между всеми вершинами \textit{полным} или же \textit{кликой} и обозначим как $K_n$
\end{definition}


Теперь перейдем к нашему первому значимуму утверждению в Теории графов


\begin{lemma}[Лемма о рукопожатиях]  
Количество вершин нечетной степени в графе четно, а сумма всех степеней равна удвоенному количеству ребер. \label{lemma1}
\end{lemma}
\begin{proof}
В сумму степеней мы учитываем каждое ребро дважды, потому что оно имеет два конца, а следовательно из четности суммы степеней вытекает и утверждение про количество вершин нечетной степени.
\end{proof}

\subsection{Подграфы}

\begin{definition}
    Назовем граф $H$ \textit{подграфом} графа $G$, если $V(H) \subset V(G)$ и  $E(H) \subset E(G)$. Если $V(H) = V(G)$ то назовем такой подграф \textit{остовным}, \textit{собственным} назовем подграф, который отличен от исходного графа G.
\end{definition}

\begin{definition}
    Пусть $U \subseteq V(G)$, тогда обозначим за $G(U)$ \textit{индуцированный} подграф на множестве вершин U. Это означает, что $V(G(U)) = U$ и $E(G(U))$ содержит ребра из $E(G)$, оба конца которых лежат в $U$.   
\end{definition}

\subsection{Операции на графах}

Теперь когда мы ввели необходимые базовые понятия о графе можно также задать несколько операций над графами.

\begin{definition}
    \textit{Объединение} графов $G_{1}$ и $G_{2}$ это такой граф $G =  G_{1} \cup G_{2}$, что $V(G) = V(G_{1}) \cup V(G_{2})$ и $E(G) = E(G_{1}) \cup E(G_{2})$ 
\end{definition}

\begin{definition}
    \textit{Дополнение} графа $G$ это такой граф $\overline{G}$, что $G \cup \overline{G} = K_{v(G)}$ (т.е. в оюъединении с исходным получается полный граф).
\end{definition}

\begin{definition}
Для любого множествa $R \subset E(G) \cup V(G)$ обозначим через $G-R$ граф, полученный из $G$ в результате удаления всех вершин и рёбер множества $R$, а также всех рёбер, инцидентных вершинам из $R$. Для $x \in E(G) \cup V(G)$ положим $G-x = G-\{x\}$.    
\end{definition}

\begin{definition}
Пусть e — ребро, соединяющее пару вершин из $V(G)$, не обязательно входящее в $E(G)$. Если $e \notin E(G)$, то через $G + e$ мы будем обозначать граф, полученный из $G$ в результате добавления ребра e (то есть, $G + e = (V (G), E(G) \cup {e})$). Если $e \in E(G)$, то $G + e = G$.
\end{definition}

\subsection{Пути и циклы. начало}

Попробуем быстро пройтись по основным определениям, которые еще понадобятся нам сегодня, а в следующий день более подробно рассмотрим эту тему.

\begin{definition}
    \textit{Маршрутом} назовем некоторую последовательность вершин $v_1v_2..v_3$ и ребер $e_1e_2..e_{n-1}$ таких что для всех индексов i: $e_i = a_ia_{i+1}$. 
    
    1) Если $v_1 = v_n$ то есть начало и конец маршрута совпадают, то маршрут называется 
    \textit{замкнутым}
    
    2) Если в маршруте не повторяются ребра, то назовем его \textit{путем}
    
    3) Если в \textit{пути} не повторяются еще и вершины, то назовем такой путь \textit{простым}
    
    4) Если \textit{путь} \textit{замкнутый} и то он называется \textit{циклом}

    5) Соответсвенно если в \textit{цикле} все вершины разные, то он также называется \textit{простым}
\end{definition}

Также для удобства будем называть путем подграф на вершинах и ребрах последовательности. 

\begin{definition}
    1) Назовем \textit{длиной} пути количество входящих в него ребер.
    
    2) Возьмем две вершины x и y и назовем \textit{xy-путем} любой простой путь с началом в x и концом в y. 
    
    3) Назовем \textit{расстоянием} между вершинами x и y величину $dist_G(x,y)$ равную длине наименьшего xy-пути.  
\end{definition}

Теперь мы готовы к доказательству чуть более содержательных утверждений.

\begin{lemma} 
    1) Для любого цикла $C$ существует такой простой цикл $C'$, что $V(C') \subset V(C)$ и $E(C') \subset E(C)$
    2) Если в графе есть нечетный цикл, то есть и простой нечетный цикл. \label{lemma2}
\end{lemma}
\begin{proof}
    1) Начнем идти по циклу $C$ и так как он замкнут мы рано или поздно должны будем прийти в уже посещенную нами вершину. Посмотрим на первую такую вершину $v$ которая встретилась нам дважды, тогда искомый цикл и будет состоять из всех вершин и ребер от первого вхождения  вершины $v$ до следующего. Простота очевидна из выбора $v$ как первой встретившейся дважды.
    2) Воспользуемся приницпом математической индукции

    \question Все ли знакомы с принципом математической индукции? (Вероятнее всего стоит объяснить)

    и предположим, что для меньших циклов мы уже доказали необходимое утверждение, базой индукции будут служить простые циклы, но для них все очевидно из условия, докажем переход. Также как в первом пункте будет искать первый простой цикл, и разомкнем цикл $C$ на два цикла  - простой $C'$ и оставшийся $C''$, с пересечением по вершине $v$, но заметим что $e(C) = e(C') + e(C'')$, если первое слагаемое нечетно - мы нашли искомый цикл, иначе из предположения такой цикл существует в оставшемся графе $C''$.
\end{proof}


\begin{lemma} 
    1) В графе есть простой путь длины хотя бы $\delta(G)$
    2) Если $\delta(G) \ge 2$, то есть также простой цикл длины хотя бы $\delta(G) + 1$ \label{lemma3}
\end{lemma}
\begin{proof}
    1) Рассмотрим путь $P = v_1v_2..v_n$ такой что длина его максимальна. Из условия мы знаем, что вершина $a_n$ должна иметь еще хотя бы $\delta(G) - 1$ соседа, но заметим, что все ее соседи также должны входить в путь P иначе мы бы смогли сделать его длиннее. Но тогда имеем $n - 2 \ge \delta(G) - 1 \implies n - 1 \ge \delta(G)$ что и есть длина пути. 

    2) Пускай теперь $v_m$ - вершина пути смежная с $v_n$ такая что ее номер наименьший, но тогда множество вершин $\{a_m, a_{m+1}, .. a_{n-1} \}$ должно содержвть хотя бы $\delta(G)$ концов ребер выходящих из $a_n$. Т.к. $\delta(G) \ge 2$ получаем что $a_m \not= a_n$ Следовательно цикл $a_ma_{m+1}..a_{n-1}a_n$ содержит хотя бы $\delta(G) + 1$ вершину
\end{proof}


\subsection{Связность. начало}

\begin{definition}
    Назовем вершины $u, v$ \textit{связанными} если в графе существует путь между ними. Назоыем граф $G$ \textit{связным} если любые две его вершины \textit{связаны}. Если граф $G$ не связен будем называть \textit{компонентами связности} его максимальные по включению связные подграфы, а количество компонпент связности обозначим за $c(G)$.
\end{definition}

\question Могут ли компоненты связности пересекаться?

\question Знакомо ли кому-то отношение сравнения по модулю и видите ли вы какую-либо связь?*

\subsection{Двудольные графы}

\begin{definition}
    Граф называется \textit{двудольным}, если его вершины можно правильно разбить на два множества (или как еще говорят покрасить в два цвета, но об этом позже), внутри которых нет рёбер (эти множества называются \textit{долями}). 

    Через $K_{n,m}$ обозначим \textit{полный двудольный} граф, доли которого содержат m и n вершин и проведены все ребра между долями.
\end{definition}

 \begin{theorem}[Критерий двудольности]
     Граф является двудольным тогда и только тогда, когда он не содержит циклов нечетной длины. 
 \end{theorem}
\begin{proof}

    $\implies$ : Если граф двудольный, то у него нет циклов нечетной длины. Рассмотрим двудольный граф. Начнем цикл в доле $U$. Нужно пройти по четному числу ребер, чтобы вернуться в $U$ снова. Следовательно, при замыкании цикла число ребер будет четным.
    
    $\impliedby$ : Если граф не имеет циклов нечетной длины, то он двудольный.
    Пусть граф $G$ связен, иначе применим рассуждения отдельно для каждой компоненты.
    Выберем произвольно вершину $u$ и разобьем множество всех вершин на два непересекающихся множества $U$ и $V$ так, чтобы в $U$ лежали вершины $v_1$, такие что $dist_G(u,v_1)$ была чётной длины, а в $V$ соответственно вершины $v_2$, для которых $dist_G(u,v_2)$ — нечётная. При этом $u \in U$. ЕСли нам удасться доказать, что в графе $G$ нет ребер ab, таких что a,b лежат одновременно в $U$ и $V$, то мы докажем нужное утверждение. Докажем это от противного.
    Н.У.О. Пусть $a,b \in U$. Пусть $P_1$ - кратчайший ua-путь, а $P_2$ - кратчайший ub-путь. Посмотрим последнюю вершину v пути $P_1$, такую что она принадлежит пути $P_2$. Тогда пути от u до v в $P_1$ и $P_2$ имеют одинаковую длину (иначе бы, пройдя по более короткому пути от u до v мы смогли бы найти более короткий путь от u до a или от u до b, чем $P_1$ или $P_2$ ). Так как пути от v до a и от v до b в $P_1$ и $P_2$ имеют одинаковую четность, в сумме с ребром ab они образуют цикл нечётной длины, но это невозможно.
\end{proof}

\section{День 2. Деревья, пути, циклы}

\subsection{Деревья}

\begin{definition}
    \textit{Дерево} это связный граф без циклов. \textit{Лес} это граф без циклов. Вершину графа $G$ имеющую степень 1 назовем \textit{висячей} или же \textit{листом}. 
\end{definition}

\question Что представляют из себя компоненты связности Леса ?

\begin{theorem}
    В дереве на n вершинах в точности n - 1 ребро. Также у любого связного графа существует \textit{остовное дерево}, то есть дерево представялет из себя минимальный по количеству ребер связный граф без циклов. 
\end{theorem}

\begin{proof}
Докажем индукцией по количеству вершин в дереве. База индукции для дерева с одной вершиной очевидна. 

Рассмотрим дерево $T$ с $n \ge 2$ вершинами. По Лемме \Cref{lemma3} в графе, степени всех вершин которого не менее 2, есть цикл. Очевидно, у связного графа $T$
на $n \ge 2$ вершинах не может быть вершин степени 0. Значит, у дерева $T$
есть висячая вершина a. Понятно, что граф $T - a$ также связен и не
имеет циклов, то есть, это дерево на $n - 1$ вершинах. По индукционному
предположению мы имеем $e(T - a) = n - 2$, откуда очевидно следует,
что $e(T) = n - 1$.

 Если в графе есть цикл, то можно удалить из этого цикла ребро.
Граф, очевидно, останется связным. Продолжим такие действия до тех
пор, пока циклы не исчезнут. Понятно, что рано или поздно это произойдет, так как с каждым шагом уменьшается количество рёбер, а оно
изначально конечно. В результате мы получим связный граф без циклов,
являющийся остовным подграфом исходного графа, то есть, остовное дерево этого графа.
\end{proof}

 \question Как в таком случае можно оценить снизу количест во ребер в любом графе, в том числе и не связном?

\begin{consequence}
    Дерево более чем с одной вершиной имеет хотя бы два листа
\end{consequence}
 \begin{proof}
    Пусть в дереве не более 1 листа, тогда из связности имеем, что степени всех вершин хотя бы 2, тогда сумма степенией не менее $2(v(G) - 1) + 1 = 2v(G) - 1$, но с другой стороны по ЛЕмме о рукопожатиях сумма степеней это в точности удвоенное количество ребер, из теоремы выше имеем, что $2e(G) = 2(v(G) - 1)$ то есть $2v(G) - 2 \ge 2v(G)  - 1$. Противоречие
 \end{proof}

\begin{theorem}
    Граф является деревом тогда и только тогда, когда между любыми двумя вершинами существует только единственный простой путь, который их соединяет.
\end{theorem}

\begin{proof}

    $\implies$ : Если граф дерево, то между любыми двумя его вершинами существует ровно один простой путь. Хотя бы один путь между любыми веришинами существует из условия связности графа $G$. Пускай существует два различных ab-пути $P_1$ и $P_2$. Пройдем первые общие ребра этих путей, пока не встретим расхождение в вершине c, затем продолжим идти по пути $P_1$ до первого пересечения с путем $P_2$ в некоторой веришне d, очевидно такая найдется, потому что пути имеют общий конец b. Тогда мы получили 2 простых cd-пути без общих вершин, которые очевидно образуют цикл, но тогда $G$ не является деревом.
    
    $\impliedby$ : Если между любыми двумя вершинами графа существует ровно один простой путь, то граф - дерево. Очевидно, граф будет связен, тогда предположим он не является деревом и у него существует цикл, не трудно понять что в таком случае мы получаем противоречие с единственностью пути между любыми двумя вершинами т.к. в имеющемся цикле можно изменить направление обхода и получить два пути между вершниами. Следовательно наш граф  - дерево.

\end{proof}

\subsection{Дерево обхода в ширину}

Пусть дан связный граф $G$ будем говорить, что он \textit{подвешен} за некоторую вершину a, если существует некоторое разбиение множетсва $V(G)$ на семейство непересекающихся множеств $L_i$ таких что в $L_i$ содержаться все вершины $l$ удовлетворяющие $dist_G(a, l) = i$. Назовем множество $L_i$ \textit{уровнем i}, очевидно уровень 0 состоит только из вершины a, назовем ее \textit{корнем}, затем вершины уровня i присоединяются ребрами к одной из смежных вершин уровня i - 1. Очевидно мы получили дерево (т.к. каждый раз добавляли ровно 1 вершину и 1 ребро), да и к тому же остовное, назовем его \textit{Деревом обхода в ширину с корнем в a}. 

Стоит сделать важное замечание: полученное нами разбиение на уровни единственное для выбранного корня a, но само дерево нет.

\begin{lemma} 
        Ребра исходного графа $G$ могут соединять либо вершины соседних уровней, либо вершины того же уровня.\label{lemma4}
\end{lemma}
\begin{proof}
    Действительно пусть существует ребро $xy \in E(G)$ такое что x из уровня k, а y из уровня m, таких что $m > k$, тогда имеем 
    $$m = dist_G(a,y) \le dist_G(a,x) + 1 = k+1$$  
\end{proof}

Если кто-то из вас знаком с алгоритмом обхода в ширину BFS (breadth-first-search), то полученный нами граф является непосредственной материализацией данного алгоритма.

\subsection{Дерево обхода в глубину}

\begin{definition}
    Пусть $G$ — связный граф, $a \in V(G)$. Остовное дерево $T$ называется \textit{Деревом обхода в глубину с корнем в a} или \textit{нормальным деревом с корнем a}, если для любого ребра $xy \in E(G)$ либо x лежит на ay-пути дерева $T$, либо y лежит на ax-пути дерева $T$.
\end{definition}

В отличие от конструктивно полученной предыдущей структуры, существование нормальных деревеьев требует доказательства.

\begin{theorem}
    Пусть дан связный граф $G$ и некоторая вершина $a \in V(G)$. Тогда у графа существует $T$ -  Дерево обхода в глубину с корнем в a.
\end{theorem}

\begin{proof}
    Будем вести индукцию по количеству вершин. В качестве базы выступают графы с 1 или вершинами, для которых все очевидно ($T$ = $G$). 
Докажем переход индукции:  рассмотрим множество $\{G_1,..G_m\}$ компонент связности графа $G - a$. В каждой компоненте отметим вершину $a_i \in U_i \cap N_G(a)$ и построим с помощью предположения индукции дерево обхода в глубину $T_i$ для графа $G_i$ с корнем в $a_i$. Поcле этого соединим все $a_1,..a_m$ с корнем a. Утверждается, что мы получили остовное дерево $T$ удовлетворяющее условию нормальности. 

Проверим данное утверждение. Пусть $xy \in E(G)$. Если обе вершины x и y отличны от a, то они лежат в одной из компонент связности $U_i$ (так как рёбер между разными компонентами нет), а значит, свойство для ребра xy выполнено по индукционному предположению для $T_i$ ( если, скажем, x лежит на $a_iy$-пути по $T_i$,  то x лежит и на ay-пути по T) . Если же x = a, то доказываемое свойство для ребра xy очевидно.
\end{proof}

Аналогично предыдущей структуре Дерево обхода в глубину является материализацией алгоритма DFS (depth-first-search), а если точнее то полученное в результате работы алгоритма дерево рекурсии будет нормальным.

\subsection{Эйлеров цикл}

В этой главе мы допустим существование кратных ребер.

\begin{definition}
    Назовем путь в графе $G$ \textit{эйлеровым} если он проходит по каждому ребру в точности один раз. Аналогично введем \textit{эйлеров цикл}. Назовем граф $G$ \textit{эйлеровым} если у него есть эйлеров цикл.
\end{definition}

\begin{theorem}
    Граф $G$ является эйлеровым если и только если он связен и степени всех его вершин четны.
\end{theorem}

\begin{proof}
    
    $\implies$ : Если граф эйлеров, то все его вершины имеют четную степень. Пусть изначально все вершины имеют степень 0. При обходе графа по эйлерову циклу мы пройдем по каждому ребру единожды, а каждый раз посещая очередную вершину, кроме первой, ее степень будет увеличиваться на 2 (зашли в вершину и вышли по другому ребру), отчего будет поддержан инвариант четности степеней, для первой же степень будет оставаться нечетной пока мы не вернемся в нее для замыкания цикла, прибавив 1, очевидно из конечности количества ребер алгоритм когда-нибудь завершится.
    
    $\impliedby$ : Если все вершины графа имеют четную степень, то он эйлеров. 
    Заметим что т.к. все степени хотя бы 2, то граф не может быть деревом, значит у него существует цикл, а значит по Лемме \Cref{lemma2} существует и простой цикл. 
    
    Будем вести индукцию по количеству простых циклов в графе. Базой индукции будет служить ситуация когда граф сам по себе является простым циклом, тогда все очевидно этот цикл и будет эйлеровым.
    
    Начнем путь в произвольной вершине a и будем идти и будем строить подграф $C$. Так как все степени четны, то наш путь обязательно закончится в вершине a. В результате $C$ циклом. Рассмотрим компоненты связности на графе $G - E(C)$. В каждой компоненте очевидно уменьшилось количество простых путей, а также степени всех вершин остались четными ( при прохождении если мы вошли в какую-то вершину, то мы обязаательно из нее выйдем итого степень уменьшается на 2 ), тогда мы в праве применить предположение индукции для этих подграфов и найти в них эйлеровы циклы $C_i$. Из связности исходного графа очевидно существование вершины $v_i \in V(C_i) \cap V(C)$ тогда поочередно будем стыковать эйлеровы циклы компонент в цикл C так: обойдем цикл C от a до $v_i$ затем обойдем всю компоненту циклом $C_i$ и по циклу C вернемся в a. Таким образом будет получен Эйлеров цикл графа $G$. 
\end{proof}

\begin{consequence}
     Граф $G$ имеет эйлеров путь если и только если он связен и либо все его степени четны, либо вершин нечетной степени ровно две.
\end{consequence}

\begin{proof}
        
    $\implies$ : Если граф имеет эйлеров путь, то верно условие на степени. Пусть эйлеров путь имеет концы a и b. Если $a = b$, то наш путь — эйлеров цикл, а значит, степени всех вершин четны. Если же $a \not= b$, то рассмотрим граф $G + ab$ (если такое ребро есть, то добавим еще одно), он очевидно имеет эйлеров цикл по теореме выше, значит все степени четны, значит в исходном графе ровно две вершины  - a и b - имеют нечетную степень.
    
    $\impliedby$ : Если верно условие на степени, то граф имеет эйлеров путь. 
    Если вершин нечетной степени нет, то в графе есть эйлеров цикл, который является и эйлеровым путем. Пусть в графе $G$ ровно две вершины нечетной степени a и b. Добавим
    в граф ребро ab (если такое ребро есть, то добавим еще одно). Мы получили связный граф $G'$, в котором все вершины имеют четную степень и по теореме выше есть эйлеров цикл $C$. Удалим из цикла $C$ добавленное ребро ab и получим эйлеров путь с концами a и b.
\end{proof}

\subsection{Гамильтонов цикл}

В этой главе мы уже запретим возможность граф иметь петли и кратные ребра.

\begin{definition}
    Назовем \textit{гамильтоновым} простой путь в графе $G$ который проходит по каждой вершине графа в точности один раз.  Аналогично определим цикл. Назовем граф $G$ \textit{гамильтоновым}, если у него есть \textit{гамильтонов цикл}.
\end{definition}

\label{lemma5}
\begin{lemma} 
    Пусть $n > 2$, $v_1..v_n$ - максимальный путь в графе $G$, причем $d_G(v_1) + d_G(v_n) \ge n$ Тогда граф имеет цикл длины n. 
\end{lemma}
\begin{proof}
    Если вершины $v_1$ и $v_n$ смежны, то из $n > 2$ следует, что в графе существует цикл длины n. Пусть тогда они несмежны, посмотрим на окрестности этих вершин, понятно, что 
    $$
    N_G(v_1), N_G(v_n) \subset \{v_2,..v_{n-1}\}
    $$
    Т.к. иначе мы бы смогли продлить путь.
    Рассмотрим два случая:
    1) Существуют такие вершины $v_k \in N_G(v_n)$ и $v_{k+1} \in N_G(v_1)$ (то есть смежные в пути), тогда мы сможем построить цикл длины n вида $v_1v_2..v_kv_nv_{n-1}v_{n-2}..v_{k+1}$.
    2) Таких вершин не существует то есть для любой вершины 
    $$v_i \in N_G(v_n) \implies v_{i+1} \notin N_G(v_1)$$
    $$ \implies ( |N_G(a_n)| = d_G(a_n) ) d_G(v_1) \le n - 1 - d_G(a_n) $$
    $$ \implies d_G(v_1) + d_G(v_n) < n$$
    Что противоречит условию.
\end{proof}


\begin{theorem}[Ø.Ore 1960]
    Если для любых двух несмежных вершин $u,v \in V(G)$ выполняется:
    
    1) $d_G(u) + d_G(v) \ge v(G) - 1$ то в графе есть гамильтонов путь 
    
    2) $d_G(u) + d_G(v) \ge v(G)$ то в графе есть гамильтонов цикл 
\end{theorem}
\begin{proof}
    1) Случай, когда в графе $G$ ровно две вершины очевиден. Теперь пусть $v(G) > 2$ 
    Докежем что $G$ обязательно связен. Пускай a и b - две несмежные вершины графа, тогда из $d_G(u) + d_G(v) \ge v(G) - 1$ и принципа Дирихле следует, что $N_G(a) \cap N_G(b) \not= \varnothing$.
    Путь $n < v(G)$ - количество вершин в наибольшем простом пути графа $G$. Поскольку граф $G$ связен и $v(G) > 2$, то $n \ge 3$. По Лемме \Cref{lemma5} в графе $G$ есть цикл $C$ из n вершин. Так как граф связен, существует не вошедшая в этот цикл вершина, смежная хотя бы с одной из вершин цикла, но тогда, очевидно, существует и путь на $n + 1$ вершине, что противоречит предположению. Таким образом, в графе есть гамильтонов путь.

    2) По предыдущему пункту в графе $G$ существует гамильтонов путь $u_1u_2..u_{v(G)}$ 
    Если $u_1$ и $u_{v(G)}$ смежны, то мы нашли гамильтонов цикл, пусть не смежны, но тогда наличие гамильтонова цикла следует из Леммы \Cref{lemma5} и $d_G(u) + d_G(v) \ge v(G)$. 
\end{proof}

\begin{remark}
    Øystein Ore - норвежский математик, известный по своим работам в теории колец и графов, также был научным руководителем Маршалла Холла младшего, который в свою очередь был научным руководитеелм Дональда Кнута - одного из самых известных ученых в сфере Computer Science.
\end{remark}

\begin{consequence} [G.A. Dirac 1952]

    1) Если $\delta(G) \ge \frac{v(G) - 1}{2}$, то в графе есть гамильтонов путь

    2) Если $\delta(G) \ge \frac{v(G)}{2}$, то в графе есть гамильтонов цикл
\end{consequence}

\begin{remark}
    Gabriel Andrew Dirac - венгерско-британский математик, работавший в сфере теории графов. Его отцом является Пол Дирак, который в 1933 разделил вместе с Шредингером нобелевскую премию по физике в области квантовой механики и вообще считается одним из отцов основателей этой науки.  
\end{remark}

Можно заметить, что в отличие от условий на Эйлеровы графы, условия на Гамильтоновы графы не являются необходимыми и достаточными, и насколько мне известно, на сегодняшний день таких сильных все еще утверждений не доказано.

\section{День 3. Паросочетания}

\subsection{Независимые множества и покрытия}

Сейчас будет достаточно много определений, но все они достаточно просты в понимании, но тем не менее достаточно важны для развития темы этого дня.

\begin{definition}
    Назовем множество вершин $U \subset V(G)$ \textit{независимым} в графе $G$ если никакие две вершины в нем не смежны.
\end{definition}

\begin{definition}
    Назовем множество ребер $M \subset E(G)$ \textit{паросочетанием} в графе $G$ если никакие два ребра в нем не имеют общих вершин.
\end{definition}

\begin{definition}
    Будем говорить, что множество вершин $W \subset V(G)$ \textit{покрывает} ребро $e \in E(G)$, если существует $w \in W$ инцидентная $e$. Симметрично будем говорить, что множество ребер $F \subset E(G)$ \textit{покрывает} вершину $v \in V(G)$, если существует $f \in F$ инцидентное $v$.
\end{definition}

\begin{definition}
    Назовем паросочетание \textit{совершенным}, если оно покрывает все вершины графа.
\end{definition}

\begin{definition}
    Множество вершин $W \subset V(G)$ назовем \textit{вершинным покрытием}, если оно покрывает все ребра графа $G$. Симметрично множество ребер $F \subset E(G) $ назовем \textit{реберным покрытием} если оно покрывает все вершины графа $G$.
\end{definition}

    Теперь для простоты дальнейшего изложения введем обозначения для определений выше: 
    
    1) $\alpha(G)$ - \textbf{максимальное} \textit{независимое множество} 

    2) $\alpha'(G)$ - \textbf{максимальное} \textit{паросочетание} 

    3) $\beta(G)$ - \textbf{минимальное} \textit{вершинное покрытие} 

    4) $\beta'(G)$ - \textbf{минимальное} \textit{реберное покрытие} 

    
А также докажем некоторые соотношения между этими величинами

\begin{lemma} 
      $U \subset V(G) $ - независимое множество, если и только если $V(G) \setminus U$ - вершинное покрытие. 
      
      Кроме того $\alpha(G) + \beta(G) = v(G)$ . \label{lemma6} 
\end{lemma}
\begin{proof}

    $\implies$ : Если $U$ независимое множество, то $V(G) \setminus U$ - вершинное покрытие.
    Пусть это не так, следовательно существует ребро, которое оказалось не покрыто  $V(G) \setminus U$, но тогда оба его конца должны принадлежать $U$, что противоречит независимости множества.

    $\impliedby$ : Если $V(G) \setminus U$ вершинное покрытие, то  $U$ независимое множество. Пускай $U$ не является независимым, тогда существует ребро, оба конча которого входят в $U$, но тогда оно не может быть покрыто $V(G) \setminus U$. Противоречие.

    $U$ -  максимальное независимое множество, если и только если $V(G) \setminus U$ - минимальное покрывающее множество.
\end{proof}

\subsection{Теорема Галлаи}

\begin{theorem} [T. Gallai 1959]
    Пусть $G$ - граф такой что $\delta(G)  > 0$. Тогда  $\alpha'(G) + \beta'(G) = v(G)$
\end{theorem}

\begin{proof}
    Докажем, что  $\alpha'(G) + \beta'(G) \le v(G)$: 
    
    Пусть M - максимальное паросочетание, а U - множество непокрытых M вершин графа G, тогда $|U| = v(G) - 2\alpha'(G)$
    Т.к. $\delta(G) > 0$ мы сможешь выбрать такое множество $F \subset E(G)$, что $|F| = |U|$ и F покрывает  U. Следовательно $M \cup F$ - покрытие, следовательно 
    $\beta'(G) \le |M \cup F| = \alpha'(G) + v(G) - 2\alpha'(G)$, откуда получаем необходимое неравенство.
    
    Теперь докажем $\alpha'(G) + \beta'(G) \ge v(G)$:

    Пусть L - минимальное реберное покрытие ($|L| = \beta'(G)$), а \\ $H = (V(G), L)$.
    Т.к. в графе H нет вершин степени 0, в каждой его компоненте можно выбрать по ребру, в результате получится паросочетание N в графе H (а значит и в G), следовательно: 
    $$
    \alpha'(G) \ge |N| = c(H)
    $$
    $$
    \implies \beta'(G) = |L| = e(H) \ge v(H) - c(H) \ge v(G) - \alpha'(G)
    $$
    $$
    \implies \alpha'(G) + \beta'(G) \ge v(G)
    $$
\end{proof}

\begin{remark}
    Tibor Gallai - венгерский математик, достиг достаочно большого числа результатов в теории графов, друг и товарищ Пола Эрдёша.
\end{remark}

\subsection{Теорема Бержа}

\begin{definition}
    Пусть M — паросочетание в графе G. Назовём путь \textit{M-чередующимся}, если в нём чередуются рёбра из M и рёбра, не входящие в M. Назовём M-чередующийся путь \textit{M-дополняющим}, если его начало и конец не покрыты паросочетанием M.
\end{definition}

Отметим, что в любом M-дополняющем пути нечётное число рёбер и
чётное число вершин.

\begin{theorem} [C.J. Berge 1957]
    Паросочетание M в графе G является максимальным тогда и только тогда, когда нет M-дополняющих
путей.
\end{theorem}
\begin{proof}

$\implies$ Если парасочетание максимальное, то у него нет дополнящию путей. 
Пусть в графе существует M-дополняющий путь $ S = v_1v_2..v_{2k}$,  тогда заменим входящие в M ребра $v_2v_3 , ..v_{2k-2}v_{2k-1}$ на невходящие $v_1v_2, v_3v_4, .. v_{2k-1
}v_{2k}$, при этом мы только увеличили паросочетание, тем самым получив противоречие с максимальностью M.


$\impliedby$ Если у парасочетания нет дополняющего пути, то оно максимальноею 
Пусть M - не максимальное паросочетание, тогда рассмотрим максимальное парасочетание M', |M'| > |M|. Пусть $N = M \triangle M', H = G(N)$. Тогда для любой $v \in V(H)$ мы имеем $d_H(v) \in \{1,2\}$. Следовательно H - это объединение нескольких путей и циклов.. Очевидно, в каждом
из этих путей и циклов рёбра паросочетаний M и M' чередуются. Так
как рёбер из M' в E(H) больше, хотя бы одна компонента P графа H — путь нечётной длины, в котором больше рёбер из M'. Легко понять, что P — это M-дополняющий путь.
\end{proof}

\begin{remark}
    Claude Jacques Berge - французский математик, работал в области комбинаторики и теории графов.
\end{remark}

\subsection{Теорема Холла}

Теперь докажем одину из самых часто встречающихся и известных теорем в теории графов.

\begin{theorem} [P. Hall 1935]
В двудольном графе G есть паросочетание покрывающее все вершины доли $V_1(G)$ тогда и только тогда, когда для любого $U \subset V_1(G)$ выполнено $|U| \le |N_G(U)|$. (условие на окрестность для доли $V_1(G)$ будем называть \textit{условием Холла}).
\end{theorem}
\begin{proof}
    $\implies$ : Если существует паросочетание покрывающее долю $V_1(G)$, то выполнено условие Хлолла. 
    Оставим читателю в качестве упражнения.

    $\impliedby$ : Если выполнено условие Холла, то существует паросоетание покрывающее долю $V_1(G)$.
    Будем вести индукцию по количеству вершин в графе. База для $|V_1(G)| = 1$ представляется очевидной. Тогда предположим что для меньших графов все доказано тогда разберем два случая: 


    1) Существует непустое собственное подмножество $A \subset V_1(G)$, что $|A| = |N_G(A)|$

    Для удобства обозначим $B = N_G(A)$, $A' = V_1(G) \setminus A$, $B' = V_2(G) \setminus B$. 
    $G_1 = G(A \cup B) , G_2 = G(A' \cup B')$
    Тогда для графа $G_1$ и доли A будет выполнено условие Холла, а также он будет меньше по количеству вершин , следовательно по предположению индукции существует паросочетание $M_1$ покрывающее A.
    
    Остается проверить условие Холла для графа $G_2$ и доли A'. Рассмотрим $U \subset A'$: 
    $$
    |U| + |A|  = |U \cup A| \le |N_G(U \cup A)| = |N_{G_2}(U) \cup B| = |N_{G_2}(U)| + |B| = 
    $$
    $$
     |N_{G_2}(U)| + |A| \implies |U| \le |N_{G_2}(U)|
    $$

    А значит существует паросочетание $M_2$ покрывающее A', а тогда паросочетание $M_1 \cup M_2$
    покрывает $V_1(G)$ 

    2) Для любого непустого собственного подмножества $A \subset V_1(G)$ выполняется $|A| < |N_G(A)|$.

    Рассмотрим произвольную вершину $a \in V_1(G)$ и смежную с ней вершину $b \in V_2(G)$
    Рассмотрим граф $G' = G - a - b$ и проверим для него и его доли $V_1(G) \setminus \{a\}$ условие Холла. Для любого множества $A \subset V_1(G) \setminus \{a\}$ верно:
    $$
    |A| < |N_G(A)|  \implies |A| \le  |N_G(A)| - 1 \le |N_G(A) \setminus \{b\}| = |N_{G'}(A)| 
    $$
    Тогда в графе G' существует паросочетание M' покрывающее долю $V_1(G) \setminus \{a\}$, а тогда $M' + ab$ искомое паросочетание покрывающее $V_1(G)$
\end{proof}

\begin{remark}
    Можно дать в качестве домашнего упражнения доказательство Теоремы Холла с помощью метода дополняющих путей.
\end{remark}

\begin{remark}
    Philip Hall - британский математик, занимавшийся на самом деле в основном теорией групп, изначально теорема сформулирована в двух эувивалентных формулировках комбинаторной в терминах множеств и отображений и теорграфовой, в варианте Филиппа Холла некоторое семейство множеств должно было быть обязательно конечно, однако через время упомянутый выше Маршалл Холл Младший (по иронии судьбы просто однофамилец) расширил теорему и на бесконечный случай.
\end{remark}

\subsection{Теорема Кёнига}

\begin{theorem} [D Kőnig 1931]
    Пусть G - двудольный граф. Тогда $\alpha'(G) = \beta(G)$
\end{theorem}

\begin{proof}
    \textbf{TBD}
\end{proof}

\begin{remark}
    Dénes Kőnig - венгерский математик, один из основателей теории графов как области математики, оказал сильное влияние на многие светила теории графов, многие из которых упомянуты и в этом тексте (Эрдёш, Галлаи, Туран). 
\end{remark}

\subsection{Теорема Гейла-Шэпли}

В разнообразных приложениях часто встречается случай, когда нам важно с какими вершинами строить паросочетнание. 

\begin{definition}
    1) Пусть для каждой вершины $v \in V(G)$ задан некоторый \textit{порядок} на множестве инцидентных ей ребер (то есть некоторые ребра будут более желанными для построения, нежели другие) $\le_v$. Тогда $\le = \{\le_v\}_{v \in V(G)}$ - \textit{множество предпочтений}.

    2) Паросочетание M называется \textit{стабильным} для множества предпочтений $\le$, если 
    для любого ребра $f \notin M$ существует такое ребро $e \in M$, что e и f имеют общий конец  v  и $f \le_v e$ 
\end{definition}

То есть каждая вершина имеет список предпочтений, упорядочивает инцидентные ей рёбра. Наша задача — построить такое паросочетание M (не обязательно максимальное), что в нём не будет ребра e = ab, которое обе вершины a и b хотели бы поменять на свободные рёбра.

\begin{theorem} [D. Gale, L. Shapley 1962]
    Пусть G - двудольный граф. Тогда для любого множества предпочтений $\le$ существует стабильное паросочетание. 
\end{theorem}
   
\begin{proof}
Будем считать вершины одной доли
мужчинами, а вершины другой доли — женщинами, а
наше паросочетание будет состоять из семейных пар.
Изначально наше паросочетание пусто, оно будет
изменяться пошагово. 

Опишем шаг алгоритма изменения паросочетания:

1) Сначала действуют мужчины:
каждый неженатый (то есть, не покрытый
паросочетанием) мужчина выбирает женщину, которая
ему больше всех нравится (то есть, наивысшую в своем
предпочтении) из тех, которым он еще не делал
предложения (если такие есть), после чего делает ей
предложение.

2) Затем действуют женщины:
каждая из них рассматривает всех мужчин, кто сделал ей
предложение и нравится ей строго больше, чем ее муж
(если он есть). Если это множество непусто, она
выбирает из них того, кто нравится ей больше всего
(если таких несколько, то любого из них) и выходит за
него замуж (вместо ее прежнего мужа, если он был).


Конечность алгоритма очевидна: никакой мужчина не
делает предложение одной женщине дважды. Пусть в
результате получилось паросочетнание M.

Докажем стабильность M. Рассмотрим любое ребро $mw \in E(G) \setminus M$
1) Если m делал предложение w, но либо w ему отказала,
либо сначала приняла предложение, но потом бросила, то
w нашла мужа m', который ей нравится не меньше, чем m. (то есть существует ребро $m'w \in M: mw \le_w m'w $

2) Если же m не делал предложения w, то в процессе
алгоритма нашел жену w', которая нравится ему не
меньше, чем w (то есть, существует ребро $mw' \in M : mw \le_m mw'$).

То есть ни одна из вершин ребра не хотела бы заменить его на свободное ребро, то есть построенное паросочетание M действительно стабильно
\end{proof}

\begin{remark}
     David Gale, Lloyd Shapley - американские математики и экономисты.
\end{remark}

\subsection{Теорема Татта}

\begin{definition}
    Для произвольного графа G через o(G) обозначим количество нечётных компонент связности графа G (то есть, компонент связности, содержащих нечётное число вершин).
\end{definition}

\begin{theorem} [W.T. Tutte 1947]
    В графе G существует совершенное паросочетание тогда и только тогда, когда для любого
    $S \subset V(G)$ выполняется услоиве $o(G-S) \le |S|$. (Это условие будем называть \textit{условием Татта}). 
\end{theorem}

\begin{proof}
    \textbf{TBD}
\end{proof}

\begin{remark}
    William Thomas Tutte - британский и канадский математик и криптограф. Во время второй мировой войны помог Союзникам взломать немецкий шифр Лоренца, а после оказал значимое влияние на теорию графов и матроидов.
\end{remark}

\section{День 4. Раскраски}

\subsection{Хроматическое число}

\begin{definition}
    Назовем \textit{раскраской} вершин графа $G$ в k цветов такое отображение 
    $p: V(G) \mapsto \{1..k\}$. \textit{Раскраска} называется правильной если для любой пары смежных вершин u и v выолнено $p(u) \not= p(v)$.
\end{definition}

\begin{definition}
    Через $\chi(G)$ обозначим \textit{хроматическое число} графа G — наименьшее натуральное число, для которого существует правильная раскраска вершин графа G в такое количество цветов.
\end{definition}

\begin{lemma}
    Для любого графа G верно $\chi(G)\alpha(G) \ge v(G)$  \label{lemma7}
\end{lemma}
\begin{proof}
    Все вершины одного цвета в правильной раскраске попарно несмежны, то есть образуют независимое множество.
\end{proof}

\subsection{Теорема Брукса}

\begin{lemma} \label{lemma8}
   Пусть G - связный граф, $\Delta(G) \le d$, причем хотя бы одна из
вершин графа G имеет степень менее d. Тогда $\chi(G) \le d$.
\end{lemma}
\begin{proof}
   Будем вести индукцию по количеству вершин. н. База для графа, 
у которого не более d вершин, очевидна.
Будем считать, что утверждение верно для любого меньшего
связного графа с меньшим чем v(G) количеством вершин.

Пусть $u \in V(G)$ - вершина степени менее d. Рассмотрим граф G - u, где $G_1...G_k$ его компоненты связности. В каждом из графов ввиду связности G должна существовать вершина $u_i$ смежная в графе G  с u.

Тогда $d_{G_i}(u_i) < d$ и $\Delta(G_i) \le d $. По предположению индукции существует правильная раскраска вершин графа $G_i$ в d цветов. Т.к. компоненты независимы, то такая раскраска существует и у графа G - u. А т.к. из выбора вершины u имеем $d_G(u) < d$ можно просто покрасить ее в еще один цвет не нарушая правильности покраски.
\end{proof}

\begin{theorem} [R.L. Brooks 1941]
    Доказательство теоремы Брукса, которое мне известно требует некоторых знаний о связности, которые я решил опустить, поэтому я попробую его переформулировать без этих знаний.
\end{theorem}
\begin{proof}
    \textbf{TBD}
\end{proof}

\begin{remark}
    Rowland Leonard Brooks - английский математик.
\end{remark}

\subsection{Хроматический многочлен}

\begin{definition}
    Для любого натурального k обозначим через $\chi_G(k)$ количество правильных раскрасок вершин графа G в k цветов. Такая функция называется \textit{хроматическим многочленом} графа G.
\end{definition}

\question Чему равно $\chi_G(\chi(G))$ ? 


\question Чему равно $\chi_G(k)$ если $k < \chi(G)$ ? 

\begin{definition}
    Опеделим операцию \textit{стягивания} ребра uv в графе G так: $G\cdot uv = G - u - v$, но добавим такую вершину w, что $N_G(w) = N_G(u) \cup N_G(v)$ (добавим также соответсвующие ребра).
\end{definition}

\begin{lemma} 
    Пусть G - непустой граф, а $uv \in E(G)$. Тогда 
    $$
    \chi_{G-uv}(k) = \chi_G(k) + \chi_{G\cdot uv}(k)
    $$ \label{lemma9}
\end{lemma}
\begin{proof}
    Разобьем правильные раскраски графа G - e в k цветов на два типа: те, в которых вершины u и
v одного цвета (тип 1) и те, в которых вершины u и v разных цветов (тип 2). Тогда количество раскрасок первого типа равно $\chi_{G\cdot uv}(k)$, а количество раскрасок второго типа равно $\chi_G(k)$ 
\end{proof}

\begin{theorem}
    Для графа G c $v(G) = n$ верно, что $\chi_G(k)$ - многочлен с целыми
коэффициентами степени n, старший коэффициент равен 1.
\end{theorem}
\begin{proof}
    Мы будем доказывать оба утверждения индукцией по количеству вершин и ребер графа G. А
именно, доказывая утверждение для графа G, мы будем считать его справедливым для всех меньших графов.

В качесвте базы рассмотрим пустой граф на n вершина, очевидно  $\chi_G(k) = k^n$ и все верно.

Докажем переход индукции: Пусть G — непустой граф, а e — его ребро. По
Лемме \Cref{lemma9}  $  \chi_G(k) =  \chi_{G-e}(k) - \chi_{G\cdot e}(k)$

Для меньших графов $G - e$ и $G\cdot e$ все доказано из предположения. 
$\chi_{G-e}(k)$ - многочлен степени v(G), $\chi_{G\cdot e}(k)$  - многочлен степени v(G) - 1,
а следовательно старший коэффициент исходного многочлена $ \chi_G(k) $ равен старшему коэффициенту $\chi_{G-e}(k)$ то есть 1. 
\end{proof}

\begin{theorem}
    Пусть $G_1...G_n$ - все компоненты связности G. Тогда 
    $$
    \chi_G(k) = \prod_{i = 1}^{n} \chi_{G_i}(k)
    $$
\end{theorem}
\begin{proof}
При правильной раскраске вершин графа вершины
разных компонент можно красить независимо друг от
друга. Следовательно, произведение количеств правильных
раскраскок графов $G_1...G_n$ в k цветов есть количество
правильных раскрасок вершин графа G в k цветов.
\end{proof}

\question Чему раыен хроматический многочлен в случае если  G - дерево на n вершинах?

\question Чему равен хроматический многочлен в слкчае если $G = K_n$ ? 

\subsection{Хроматический индекс}

\begin{definition}
    Назовем \textit{раскраской} ребер графа $G$ в k цветов такое отображение 
    $p: E(G) \mapsto \{1..k\}$. \textit{Раскраска} называется правильной если для любой пары смежных ребер e и e' выолнено $p(e) \not= p(e')$.
\end{definition}

Любая раскраска p ребер графа G в цвета [1..k] — это
разбиение множества E(G) в объединение
непересекающихся множеств $E_1..E_k$ где p принимает
значение i на рёбрах множества $E_i$.

Графы, рассматриваемые в этом разделе могут иметь
кратные рёбра, но не имеют петель.

\begin{definition}
    Через $\chi'(G)$ обозначим \textit{хроматический индекс} графа G — наименьшее натуральное число, для которого существует правильная раскраска вершин графа G в такое количество цветов.
\end{definition}

\subsection{Теорема Визинга}

\begin{theorem} [В.Г.Визинг 1964]
     Пусть G — граф без кратных рёбер. Тогда
     $$ \Delta(G) \le \chi'(G) \le \Delta(G) + 1$$
\end{theorem}
\begin{proof}
    \textbf{TBD}
\end{proof}

\begin{remark}
    Вадим Георгиевич Визинг - советский и украинский математик, занимался теорией графов и теорией расписаний.
\end{remark}
\section{День 5. Экстремальная теория графов}

\subsection{Числа Рамсея}

Всё началось с классической работы Рамсея (F. Ramsey) 1930 года, в которой было доказано, что в графе на достаточно большом количестве
вершин без больших клик обязательно есть большое независимое множество вершин.

Основным объектом изучения в этом разделе будут полные графы, рёбра которых покрашены в несколько цветов. Напомним, что множество
вершин, образующих полный подграф, а также сам этот подграф мы
называем кликой.

\begin{remark}
    Frank Plumpton Ramsey - британский ученый и математик, работал в области комбинаторики и логики, кроме того был приверженцем аналитической школы философии и близким другм Людвига Витгенштейна.
\end{remark}

\begin{definition}
    Пусть $m, n \in \mathbb{N}$. Число Рамсея r(m, n) — это наименьшее из всех таких чисел $x \in \mathbb{N}$, что при любой раскраске рёбер полного графа на x вершинах в два цвета обязательно найдётся клика на n вершинах с рёбрами цвета 1 или клика на m вершинах с рёбрами   цвета 2.
\end{definition}

В 1930 году Рамсей доказал, что число r(m, n) существует (то есть,
конечно). Несмотря на современные вычислительные мощности, известно немного точных значений чисел Рамсея. Очевидно,
$$
r(n, 1) = r(1, n) = 1, r(n, 2) = r(2, n) = n, r(m, n) = r(n, m).
$$

Мы приведём оценки сверху и снизу на числа Рамсея. Начнём с простейших оценок сверху.

\begin{theorem} [P. Erdős, G. Szekeres, 1935]
    Пусть $n, m \ge 2$ — натуральные числа. Тогда выполнены следующие утверждения:
    
    1) 
    $$ r(n, m) \le r(n, m - 1) + r(n - 1, m) $$
    2) Если оба числа r(n, m - 1) и r(n - 1, m) — чётные, то
    $$ r(n, m) \le r(n, m - 1) + r(n - 1, m) - 1$$
    
\end{theorem}
\begin{proof}
1) Рассмотрим клику на $r(n, m - 1) + r(n - 1, m)$ вершинах с рёбрами цветов 1 и 2 и ее произвольную вершину a. Тогда либо от вершины a отходит хотя бы r(n, m - 1) рёбер цвета 2, либо от вершины a отходит хотя бы r(n - 1, m) рёбер цвета 1. Случаи аналогичны,
рассмотрим первый случай и клику на r(n, m - 1) вершинах, соединенных
с a рёбрами цвета 2. На этих вершинах есть либо клика на n вершинах
с рёбрами цвета 1, либо клика на m - 1 вершинах с рёбрами цвета 2.
Во втором случае добавим вершину a и получим клику на m вершинах
с рёбрами цвета 2. Теперь из определения r(n, m) следует утверждение
пункта 1.

2) Рассмотрим клику на $r(n, m - 1) + r(n - 1, m) - 1$ вершинах с
рёбрами цветов 1 и 2 и её произвольную вершину a. Если вершине a
инцидентны хотя бы r(n, m - 1) рёбер цвета 2 или хотя бы r(n - 1, m)
рёбер цвета 1, то мы найдём в графе клику на n вершинах с рёбрами
цвета 1 или клику на m вершинах с рёбрами цвета 2. Остаётся лишь
случай, когда вершине a инцидентны ровно r(n, m - 1) - 1 рёбер цвета
2, то же самое для всех остальных вершин. Это означает, что в графе из
рёбер цвета 2 всего $r(n, m - 1) +r(n -1, m) - 1$ вершин и степень каждой
вершины равна r(n, m - 1) - 1. Однако, тогда в графе
нечётное количество вершин нечётной степени. Противоречие завершает
доказательство пункта 2.
\end{proof}

\begin{remark}
    Erdős Pál - один из самых продуктивных математиков за XX век, да и всю историю.  Написал более 1400 научных работ, существует даже шуточная величина характеризующая дальность ученого от Эрдёша и называющаяся число Эрдёша.
\end{remark}

\begin{consequence}
    Для натуральных чисел m, n выполняется неравенство
    $$
    r(n,m) \le \binom{n+m-2}{n-1}
    $$
\end{consequence}

\begin{proof}
    Очевидно $\binom{n+m-2}{n-1} = 1$ при n = 1 или m = 1, тогда равенство верно. Индукцией по n и m при $n, m \ge 2$ получаем
    $$
    r(n, m) \le r(n, m - 1) + r(n - 1, m) \le  \binom{n+m-3}{n-1} + \binom{n+m-3}{n-2} = \binom{n+m-2}{n-1}
    $$
\end{proof}

Теперь мы можем получить несколько точных значений чисел Рамсея. Отметим, что
$r(3, 3) \le 2r(2, 3) = 6$. Так как числа r(3, 3) и r(2, 4) четны, можно вывести неравенства $r(3, 4) \le r(3, 3) + r(2, 4) - 1 le 9$. И, наконец, $r(3, 5) \le
r(2, 5) + r(3, 4) \le 14$, а также $r(4, 4) \le 2r(3, 4) \le 18$. Все эти значения
являются точными.

\subsection{Числа Рамсея для раскрасок в несколько цветов}
    \textbf{TBD}

\subsection{Числа Рамсея больших размерностей}
    \textbf{TBD}

\subsection{Теорема Шура}

\begin{theorem} [I. Schur, 1917]
    Пусть $n \in \mathbb{N}$, а натуральные числа от
1 до n покрашены в k цветов. Тогда при любом достаточно большом n
найдется одноцветное решение уравнения x + y = z в числах \{1, . . . , n\}.
\end{theorem}

\begin{proof}
    \textbf{TBD}
\end{proof}

\begin{remark}
   Issai Schur - русский, немецкий и израильский математик, работал в области теории групп, ученик Фробениуса.
\end{remark}


\subsection{Теорема Турана}

\textbf{TBD}

\section{День 6. Зачет}


\section{Post Scriptum}

 Хотелось бы выразить особую благодарность моему преподавателю теории графов Дмитрию Валерьевичу Карпову за прекрасный курс прочитанный мне в 2022 году, а также за замечательную одноименную книгу по этой науке, несомненно эти работы глубоко легли в основу курса. А также моему бывшему студенту Михаилу Слободянюку за предоставление весьма лаконичного доказательства Теоремы Визинга.
 
\end{document}
